%**************************************%
%* Generated from MathBook XML source *%
%*    on 2016-05-02T22:09:15-06:00    *%
%*                                    *%
%*   http://mathbook.pugetsound.edu   *%
%*                                    *%
%**************************************%
\documentclass[10pt,]{article}
%% Load geometry package to allow page margin adjustments
\usepackage{geometry}
\geometry{letterpaper,total={5.0in,9.0in}}
%% Custom Preamble Entries, early (use latex.preamble.early)
%% Inline math delimiters, \(, \), need to be robust
%% 2016-01-31:  latexrelease.sty  supersedes  fixltx2e.sty
%% If  latexrelease.sty  exists, bugfix is in kernel
%% If not, bugfix is in  fixltx2e.sty
%% See:  https://tug.org/TUGboat/tb36-3/tb114ltnews22.pdf
%% and read "Fewer fragile commands" in distribution's  latexchanges.pdf
\IfFileExists{latexrelease.sty}{}{\usepackage{fixltx2e}}
%% Page Layout Adjustments (latex.geometry)
%% For unicode character support, use the "xelatex" executable
%% If never using xelatex, the next three lines can be removed
\usepackage{ifxetex}
\ifxetex\usepackage{xltxtra}\fi
%% Symbols, align environment, bracket-matrix
\usepackage{amsmath}
\usepackage{amssymb}
%% allow more columns to a matrix
%% can make this even bigger by overiding with  latex.preamble.late  processing option
\setcounter{MaxMatrixCols}{30}
%% Semantic Macros
%% To preserve meaning in a LaTeX file
%% Only defined here if required in this document
%% Subdivision Numbering, Chapters, Sections, Subsections, etc
%% Subdivision numbers may be turned off at some level ("depth")
%% A section *always* has depth 1, contrary to us counting from the document root
%% The latex default is 3.  If a larger number is present here, then
%% removing this command may make some cross-references ambiguous
%% The precursor variable $numbering-maxlevel is checked for consistency in the common XSL file
\setcounter{secnumdepth}{3}
%% Environments with amsthm package
%% Theorem-like enviroments in "plain" style, with or without proof
\usepackage{amsthm}
\theoremstyle{plain}
%% Numbering for Theorems, Conjectures, Examples, Figures, etc
%% Controlled by  numbering.theorems.level  processing parameter
%% Always need a theorem environment to set base numbering scheme
%% even if document has no theorems (but has other environments)
\newtheorem{theorem}{Theorem}[section]
%% Only variants actually used in document appear here
%% Numbering: all theorem-like numbered consecutively
%% i.e. Corollary 4.3 follows Theorem 4.2
\newtheorem{lemma}[theorem]{Lemma}
%% Definition-like environments, normal text
%% Numbering for definition, examples is in sync with theorems, etc
%% also for free-form exercises, not in exercise sections
\theoremstyle{definition}
\newtheorem{definition}[theorem]{Definition}
\newtheorem{example}[theorem]{Example}
\newtheorem{exercise}[theorem]{Exercise}
%% Localize LaTeX supplied names (possibly none)
%% Raster graphics inclusion, wrapped figures in paragraphs
\usepackage{graphicx}
%% Colors for Sage boxes and author tools (red hilites)
\usepackage[usenames,dvipsnames,svgnames,table]{xcolor}
%% More flexible list management, esp. for references and exercises
%% But also for specifying labels (ie custom order) on nested lists
\usepackage{enumitem}
%% hyperref driver does not need to be specified
\usepackage{hyperref}
%% Hyperlinking active in PDFs, all links solid and blue
\hypersetup{colorlinks=true,linkcolor=blue,citecolor=blue,filecolor=blue,urlcolor=blue}
\hypersetup{pdftitle={Connectivity}}
%% If you manually remove hyperref, leave in this next command
\providecommand\phantomsection{}
%% Graphics Preamble Entries
%% extpfeil package for certain extensible arrows,
%% as also provided by MathJax extension of the same name
%% NB: this package loads mtools, which loads calc, which redefines
%%     \setlength, so it can be removed if it seems to be in the 
%%     way and your math does not use:
%%     
%%     \xtwoheadrightarrow, \xtwoheadleftarrow, \xmapsto, \xlongequal, \xtofrom
%%     
%%     we have had to be extra careful with variable thickness
%%     lines in tables, and so also load this package late
\usepackage{extpfeil}
%% Custom Preamble Entries, late (use latex.preamble.late)
%% Begin: Author-provided macros
%% (From  docinfo/macros  element)
%% Plus three from MBX for XML characters
\newcommand{\RR}{\mathbf{R}}
\newcommand{\eps}{\varepsilon}
\newcommand{\lt}{ < }
\newcommand{\gt}{ > }
\newcommand{\amp}{ & }
%% End: Author-provided macros
%% Title page information for article
\title{Connectivity}
\date{}
\begin{document}
\thispagestyle{empty}
\maketitle
\typeout{************************************************}
\typeout{Section 1 Acknowledgement}
\typeout{************************************************}
\section[Acknowledgement]{Acknowledgement}\label{t-conn-acknowledgement}
This section is loosely inspired by Chapter 4 of Crossley's \emph{Essential Topology. You can download a \href{t-conn.pdf}{PDF print-friendly version}.}%
\typeout{************************************************}
\typeout{Section 2 Introduction}
\typeout{************************************************}
\section[Introduction]{Introduction}\label{t-conn-introduction}
A connected topological space is one that you would say consists of only one ``piece''. Of course, since we are topologists, we wish to make sense of this idea in terms of open (and closed) sets, but it is perhaps useful to explore a different characterization first.%
\par
We often say to beginning calculus students that a continuous function is one whose graph can be drawn without ``picking up the pencil''. You might try to define a connected space in a similar way.%
\begin{definition}\label{t-conn-def-path}
A \emph{path} in a topological space \(X\) is a piecewise continuous function \(\gamma \colon [0, 1] \to X\).%
\end{definition}
\par
The idea of \hyperref[t-conn-def-path]{Definition~\ref{t-conn-def-path}} is that the points \(\gamma(0)\) and \(\gamma(1)\) are the two endpoints. Since the function \(\gamma\) is continuous, the images \(\gamma(t)\) for \(0 \lt t \lt 1\) link the two endpoints without picking up the pencil. Many reasonable people would object to \hyperref[t-conn-def-path]{Definition~\ref{t-conn-def-path}}. After all, it is the \emph{image} of \(\gamma\) that we would typically call the path. However, the definition as given is ubiquitous in topology. When it is convenient we do abuse terminology and refer indirectly to the image of \(\gamma\) as a path joining \(\gamma(0)\) to \(\gamma(1)\).%
\begin{definition}\label{t-conn-def-path-connected-space}
A topological space \(X\) is \emph{path-connected} if, for every pair of points \(p\), \(q \in X\), there is a path \(\gamma \colon [0, 1] \to X\) with \(\gamma(0) = p\) and \(\gamma(1) = q\).%
\end{definition}
\begin{exercise}\label{t-conn-reverse-paths}
\hyperref[t-conn-def-path-connected-space]{Definition~\ref{t-conn-def-path-connected-space}} is symmetric in \(p\) and \(q\). That is, whenever there is \(\gamma\) connecting \(p\) and \(q\) as in \hyperref[t-conn-def-path-connected-space]{Definition~\ref{t-conn-def-path-connected-space}}, there is a path \(-\gamma\) such that \(\gamma(0) = q\) and \(\gamma(1) = p\).%
\par\smallskip
\noindent\textbf{Hint.}\hypertarget{hint-1}{}\quad
Give an explicit formula for \(-\gamma\).%
\end{exercise}
\par
The next two exercises suggest that this is a good definition. Remember that connectedness is supposed to mean that the space is ``all one piece''.%
\begin{exercise}\label{t-conn-intervals-are-path-connected}
In the topological space \(\RR_{\mathrm{std}}\), every interval is path-connected.%
\end{exercise}
\par
One generalization of the idea of ``interval'' is the notion of a convex set.%
\begin{definition}\label{t-conn-def-convex-set}
A subset \(A\) of \(\RR^n\) is \emph{convex} if for each pair of points \(p\), \(q \in A\), the segment connecting \(p\) and \(q\) is contained in \(A\).%
\end{definition}
\par
Recall also that the segment referred to in \hyperref[t-conn-def-convex-set]{Definition~\ref{t-conn-def-convex-set}} can be parametrized, in the notation of \hyperref[t-conn-def-convex-set]{Definition~\ref{t-conn-def-convex-set}}, as \(\gamma(t) = p(1-t) + qt\), for \(t \in [0, 1]\).%
\begin{exercise}\label{t-conn-convex-sets-are-path-connected}
In \(\RR^2\) with the standard topology, every \emph{convex} set is path-connected.%
\end{exercise}
\par
However, there are spaces that seem to meet our informal criterion that are not path-connected in the sense of \hyperref[t-conn-def-path-connected-space]{Definition~\ref{t-conn-def-path-connected-space}}. We need a variant of a definition from elsewhere in the book.%
\begin{definition}\label{t-conn-def-closed-topologists-sine}
The \emph{closed topologists' sine curve} is the set \[\left\{ (0, y) : -1 \leq y \leq 1 \right\} \cup \left\{ (x, \sin{\frac{1}{x}}) : 0 \lt x \lt 1 \right\}.\]%
\end{definition}
\begin{exercise}\label{t-conn-closed-topologists-sine-not-path-connected}
The closed topologists' sine curve (\hyperref[t-conn-def-closed-topologists-sine]{Definition~\ref{t-conn-def-closed-topologists-sine}}) is not path-connected.%
\end{exercise}
\par
If you disagree that this space is ``all one piece'', then tell me: when you draw it, where must you pick up the pencil?%
\par
Now that you have thought about \hyperref[t-conn-closed-topologists-sine-not-path-connected]{Exercise~\ref{t-conn-closed-topologists-sine-not-path-connected}} for a while, let me ask you not to try to continue to prove it. To prove it, we actually will need to use the more fundamental notion of \emph{connectedness}.%
\begin{definition}\label{t-conn-def-connected-space}
A topological space \(X\) is \emph{connected} if no proper subset of \(X\) is both open and closed.%
\end{definition}
\begin{lemma}\label{t-conn-recognize-disconnected}
Let \(X\) be a space. Then \(X\)is not connected if and only if it is possible to find open subsets \(U\) and \(V\) of \(X\) such that%
\leavevmode%
\begin{enumerate}
\item\hypertarget{li-1}{}\(U \cap V = \varnothing\)\item\hypertarget{li-2}{}\(U \cup V = X\)\item\hypertarget{li-3}{}Neither of \(U\) and \(V\) is empty.\end{enumerate}
\par
Such a pair of open sets is sometimes called a \emph{separation} of \(X\).%
\end{lemma}
\begin{definition}\label{t-conn-def-connected-set}
A subset \(A\) of the topological space X is connected if A is connected in the sense of \hyperref[t-conn-def-connected-space]{Definition~\ref{t-conn-def-connected-space}} when given the subspace topology.%
\end{definition}
\begin{theorem}\label{t-conn-closure-of-connected}
If \(X\) is a topological space and \(A\) is a connected subset, then \(\overline{A}\) is also connected.%
\end{theorem}
\par
The first definition, \hyperref[t-conn-def-path-connected-space]{Definition~\ref{t-conn-def-path-connected-space}}, is more intuitive, but \hyperref[t-conn-def-connected-space]{Definition~\ref{t-conn-def-connected-space}}, is more general. It therefore can capture spaces like the topologists' sine curve (\hyperref[t-conn-def-closed-topologists-sine]{Definition~\ref{t-conn-def-closed-topologists-sine}}). Let's worry about this exotic (some would say ``pathological'') example later, and focus on the familiar for now.%
\begin{example}\label{t-conn-empty-is-connected}
The empty set is a connected subset of every topological space.%
\end{example}
\begin{example}\label{t-conn-r-is-connected}
\(\RR\) (with the standard topology) is connected.%
\end{example}
\begin{example}\label{t-conn-intervals-are-connected}
Let \(a \lt b\). Then each of the intervals below is a connected subset of \(\RR\).%
\leavevmode%
\begin{enumerate}
\item\hypertarget{li-4}{}The open interval \((a, b)\)\item\hypertarget{li-5}{}The closed interval \([a, b]\)\item\hypertarget{li-6}{}The half-open interval \((a, b]\)\item\hypertarget{li-7}{}The half-open interval \([a, b)\)\end{enumerate}
\end{example}
\begin{example}\label{t-conn-euclidean-balls-are-connected}
Every open ball \(B(p, \eps) \subseteq \RR^n\) is connected.%
\end{example}
\par
The proof of the main theorem of this section, \hyperref[t-conn-continuous-image-of-connected]{Theorem~\ref{t-conn-continuous-image-of-connected}}, will follow quickly from the conceptually illuminating \hyperref[t-conn-no-surjective-map-into-0-sphere]{Lemma~\ref{t-conn-no-surjective-map-into-0-sphere}} below. The statement of \hyperref[t-conn-no-surjective-map-into-0-sphere]{Lemma~\ref{t-conn-no-surjective-map-into-0-sphere}} and the organization of our discussion around it are due to Martin Crossley's book \emph{Essential Topology}. (I will provide a proper citation at a later time.) First we need a short definition.%
\begin{definition}\label{t-conn-0-sphere}
The \emph{0-sphere} \(S^0\) is the subset \(\{-1, 1\}\) of \(\RR\) given the subspace topology.%
\end{definition}
\begin{lemma}\label{t-conn-no-surjective-map-into-0-sphere}
Let \(X\) be a connected topological space. Then there is no continuous surjective map from \(X\) to \(S^0\).%
\end{lemma}
\begin{theorem}\label{t-conn-continuous-image-of-connected}
Let \(X\) and \(Y\) be spaces, let \(X\) be connected, and let \(f \colon X \to Y\) be a continuous function. Then the image of \(f\) is connected.%
\end{theorem}
\par
%
\end{document}